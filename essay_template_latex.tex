\documentclass[conference]{IEEEtran}
\usepackage{cite}
\usepackage{amsmath,amssymb,amsfonts}
\usepackage{algorithmic}
\usepackage{graphicx}
\usepackage{textcomp}
\usepackage{xcolor}
\def\BibTeX{{\rm B\kern-.05em{\sc i\kern-.025em b}\kern-.08em
    T\kern-.1667em\lower.7ex\hbox{E}\kern-.125emX}}
    
\begin{document}
%
%   Title
%
\title{Essay Title (not the same as the paper title)\\
{\footnotesize \textsuperscript{}About the paper titled: “Every Signature is Broken: On the Insecurity of Microsoft Office’s OOXML Signatures.”
}
\thanks{}
}
%
%   Authors
%
\author{
    \IEEEauthorblockN{1\textsuperscript{st} Tygo van den Hurk}
    \IEEEauthorblockA{\textit{1705709}}
    \textit{t.j.f.h.v.d.hurk@student.tue.nl}
    \and
    \IEEEauthorblockN{2\textsuperscript{st} Aleksandr Bolotskiy}
    \IEEEauthorblockA{\textit{1567209}}
    \textit{a.bolotskiy@student.tue.nl}
    \and
    \IEEEauthorblockN{3\textsuperscript{st} Feiyang Yan}
    \IEEEauthorblockA{\textit{1812076}}
    \textit{f.yan@student.tue.nl}
}
\maketitle
%
%   Abstract
%
% \begin{abstract}
% A brief summary of the issues in this essay (that you write). The abstract should be 100-150 words long. It must summarize what “the essay” contains (and not what the paper contains), including purpose, scope and some very short conclusions. This is the preview of your essay to the reader
% \end{abstract}
% %
%   Introduction
%
\section{Introduction}
A description of the of the literature piece (the paper that you read) and the problem being addressed there. Relate those to the topics in the course, e.g. by explicitly indicating related pages in the book, slide numbers from the course slides or other supplementary reading material provided by the lecturers.

- Description of the paper
- Problem of the paper
- Introduce Our essays topic
- Introduce what Git and hashing is
- Refer to relevant literature by linking paper to course material
%
%   Main Section A
%
\section{Summary}
A couple of main sections: What are the main ideas and results of the paper that you read. These are the things that you will analyze in the discussion section. You are free to use subsections.

- explain that Microsoft documents are vulnerable
- Introduce types of the attacks that document is vulnerable explain how they work and how they bypass signature security

- 
%
%   Discussion
%
\section{Disussion}
Pros and cons of the work done in the paper that you read, things you learned. Relate it also with the things you learned in the course, e.g. whether the presented work has been of much actual use, whether it was superseded by the later developments etc. This has high weight in the assessment..

- Discuss the limitations of the paper. Praise it a bit.
- 

%
%   Conclusion
%
% \section{Conclusion}
%  Your group's own opinion on the analyzed material and the judgment of the paper. Be critical. How would you improve it if you were the author/researcher? This has high weight in the assessment.
%
%   References
%
\begin{thebibliography}{00}
\bibitem{b1} G. Eason, B. Noble, and I. N. Sneddon, ``On certain integrals of Lipschitz-Hankel type involving products of Bessel functions,'' Phil. Trans. Roy. Soc. London, vol. A247, pp. 529--551, April 1955.
\bibitem{b2} J. Clerk Maxwell, A Treatise on Electricity and Magnetism, 3rd ed., vol. 2. Oxford: Clarendon, 1892, pp.68--73.
\bibitem{b3} I. S. Jacobs and C. P. Bean, ``Fine particles, thin films and exchange anisotropy,'' in Magnetism, vol. III, G. T. Rado and H. Suhl, Eds. New York: Academic, 1963, pp. 271--350.
\bibitem{b4} K. Elissa, ``Title of paper if known,'' unpublished.
\bibitem{b5} R. Nicole, ``Title of paper with only first word capitalized,'' J. Name Stand. Abbrev., in press.
\bibitem{b6} Y. Yorozu, M. Hirano, K. Oka, and Y. Tagawa, ``Electron spectroscopy studies on magneto-optical media and plastic substrate interface,'' IEEE Transl. J. Magn. Japan, vol. 2, pp. 740--741, August 1987 [Digests 9th Annual Conf. Magnetics Japan, p. 301, 1982].
\bibitem{b7} M. Young, The Technical Writer's Handbook. Mill Valley, CA: University Science, 1989.
\end{thebibliography}
\vspace{12pt}
\end{document}
